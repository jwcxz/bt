\documentclass[letterpaper]{article}
\usepackage{alltt}
\usepackage{graphicx}
\usepackage{xspace}
\usepackage{color}
\definecolor{linkcolor}{RGB}{16,65,69}

\newcommand{\ttt}[1]{\texttt{#1}}
\newcommand{\projname}{\ttt{bt}\xspace}

\newenvironment{monospace}{\begin{quote}\begin{alltt}}{\end{alltt}\end{quote}}

\author{Joe Colosimo \\ \small{colosimo@mit.edu} \\ Group 6}
\date{\today}

\usepackage[colorlinks=true, linkcolor=linkcolor]{hyperref}
\title{\projname{} - A Realtime Beat Tracker \\ Implementation Status}
\begin{document}

\maketitle

\section{Current Status}

    The basic metronome design has been built and I'm currently implementing
    the beat classifier.  I've gotten the top level module to interface with
    VHDL nicely.

\section{Exploration}

    As I have mentioned, I'd like to possibly implement a more complicated beat
    classifier that is frequency-dependent instead of just looking at general
    energy.  There are a few algorithms that I have been investigating, but the
    most likely candidate is one from the Media
    Lab\footnote{\url{http://www.clear.rice.edu/elec301/Projects01/beat_sync/beatalgo.html}}.

    Given enough time, I'll probably also be looking at adjusting a variety of
    parameters to improve convergence time as much as possible.  There are a
    variety of things one can adjust and in my code, I've tried to make the
    system as parameterizable as possible.  For example,
    
    \begin{itemize}
        \item \textbf{Sample Width} --- I'm truncating sample sizes in order to
            conserve resources, but retaining more bits might improve beat
            classification.

        \item \textbf{Counter Precision} --- More bits in the metronome counter
            denominator means more accuracy.
            
        \item \textbf{Tempo Field Narrowing}  --- The frequency with which the
            metronome bank controller changes the metronome tempos directly
            impacts the convergence time.
    \end{itemize}


\end{document}
