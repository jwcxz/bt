\documentclass[letterpaper]{article}
\usepackage{alltt}
\usepackage{graphicx}
\usepackage{xspace}
\usepackage{color}
\usepackage{amsmath}
\definecolor{linkcolor}{RGB}{16,65,69}

\newcommand{\ttt}[1]{\texttt{#1}}
\newcommand{\projname}{\ttt{bt}\xspace}

\newenvironment{monospace}{\begin{quote}\begin{alltt}}{\end{alltt}\end{quote}}

\author{Joe Colosimo \\ \small{colosimo@mit.edu} \\ Group 6}
\date{\today}

\usepackage[colorlinks=true, linkcolor=linkcolor]{hyperref}
\title{\projname{} - A Realtime Beat Tracker \\ Further Improvements}
\begin{document}

\maketitle

\section{Status}

    As discussed in the last update, the main task for improving the beat
    tracker was centered around getting the beat classifier to be more
    accurate.  I have already made improvements to the metronome to make the
    interface use a fixed tempo and also to dampen the phase-locking a bit.
    I'm currently working to get the variance classifier correct.


\section{Variance-Sensitive Beat Classifier}
    
    If we store about 1 second worth of energies, we can perform the following
    linear regression to determine a good fit for the ratio of the current
    instantaneous energy to the last second of average energy in order to
    detect a beat.  Thus, we're looking for:

    \begin{align}
        \frac{e}{\mu_E} \geq 1.5142857 - 0.0025714 \sigma^2_E
    \end{align}

    where $\mu_E$ is the average energy, $e$ is the current instantaneous
    energy, and $\sigma^2$ is the variance of the last second's worth of
    energies.  More specifically:

    \begin{align}
        \mu_E &= \frac{1}{n} \sum_{i=0}^{n-1} e_i \\
        \sigma^2_E &= \mu_{E^2} - (\mu_E)^2
    \end{align}

    We can rearrange these equations to be more suitable for implementation in
    a digital system:

    \begin{align}
        e \geq & 1.5142857 \mu_E  - 0.0025714 \sigma^2_E \mu_E \\
          \geq & 1.5142857 \mu_E
                 - 0.0025714 (\mu_{E^2} - (\mu_E)^2) \mu_E \\
          \geq & 1.5142857 \mu_E
                 - 0.0025714 \mu_E \mu_{E^2} 
                 + 0.0025714 (\mu_E)^3 \\
          \geq & 1.5142857 \frac{1}{n} \sum_{i=0}^{n-1} e_i \\
               & - 0.0025714 \left[
                                \left(\frac{1}{n} \sum_{i=0}^{n-1} (e_i)^2\right) +
                                \left(
                                    \left(\frac{1}{n} \sum_{i=0}^{n-1} e_i\right) \cdot
                                    \left(\frac{1}{n} \sum_{i=0}^{n-1} e_i\right)
                                \right)
                             \right] \cdot
                             \left(\frac{1}{n} \sum_{i=0}^{n-1} e_i\right)
    \end{align}


    So, if we keep a shift register of the last $n$ energies, as well as the
    last $n$ energies squared, as well as a running sum of those two, then we
    can simply subtract the oldest value and add the newest value with each new
    input energy.  This allows us to quickly compute variance with just one
    extra cycle.  From there, we can use Bluespec's \ttt{FixedPoint} library to
    compute the value of $e$ in two more cycles (I'm limiting one
    multiplication per cycle due to resource constraints on the FPGA.  We then
    can compare $e$ to the current instantaneous energy to see if a beat event
    occurred or not.

    The current design does this in 4 linearly-pipelined stages.

    I'm still working on getting the algorithm to function correctly.  There's
    an error in the mathematical implementation.  I was surprised (and happy)
    to see that I can actually perform all these calculations with full
    numerical precision, though I have a mechanism in place to truncate results
    at certain stages to reduce bit-width growth.

\end{document}
